\chapter{Results}
\label{ch:chapter_4}

%% The following annotation is customary for chapter which have already been
%% published as a paper.
%\blfootnote{Parts of this chapter have been published in Annalen der Physik \textbf{324}, 289 (1906) \cite{Einstein1906}.}

%% It is only necessary to list the authors if multiple people contributed
%% significantly to the chapter.
%\authors{Albert {\titleshape Einstein}}

%% The '0pt' option ensures that no extra vertical space follows this epigraph,
%% since there is another epigraph after it.
%\epigraph[0pt]{
%    "quote1"
%}{attribution}

\begin{abstract}
This chapter commences with the introduction of a two-site model for a single thiolated Arylethynylene molecule with a 9,10-diHydroanthracene
core (`AH' molecule) in a molecular junction, both with and without spin. I find that the density matrix is in reasonable agreement with simple Boltzmann statistics, an approximation I adopt for most results. I first look at the effect of interaction on the transmission spectrum, and compare the spinless and spinfull transmissions. I then look at some stability diagrams, finding clear Coulomb diamonds whose properties I explore briefly. I find that the peak current behaves as an inverse second-order polynomial in the capacitive interaction strength $U$. I then explore the experiments by Ref.~\cite{perrinnano} and find that quantitative agreement is greatly improved by the incorporation of Coulomb Interaction. Finally, I show that a self-consistent calculation for the density matrix does improve on the qualitative agreement.
\end{abstract}

%% Start the actual chapter on a new page.
\newpage
\section{Introduction}
While I have found the rather elegant many-body Green's Function (equation~\ref{eq:mbgfresult}), this does not show the exact predictions, as these largely depend on the particular system or molecule.

In \citet{perrinnano}, the authors found a pronounced Negative-Differential Conductance (NDC), which is readily explained from a two site model that I present here. Ref.~\cite{perrinnano} confirmed the two-site model by use of DFT calculations, noting that the two sites correspond to the sum and difference of the HOMO and HOMO$-1$, resulting in two Localised Molecular Orbits (LMO). While the NDC was qualitatively explained by the two-site model, they used a prefactor $7.2 \times 10^{-5}$ to match the absolute values of the current. This has yet to be explained, and the Coulomb interaction was my prime suspect. For instance, Figure~S.4a of the supplementary information of Ref.~\cite{perrinnano} shows a clear plateau in the current around zero bias. The explanation of the NDC lies in the fact that the bias window usually encompasses only a single transmission peak, which is suppressed due to the Stark effect\cite{perrinnano}. As a result, the current first peaks before it is suppressed, leading to the NDC. Note that if no peak is in the bias-window at $V\ll1$, the current shape will have a plateau around $V=0$, which extends until a transmission peak enters the bias window. 

In section~\ref{sec:twosite}, I define the model both with and without spin. I look at the density matrix under the approximation that the many-body states are Boltzmann distributed in section~\ref{sec:dmapprox}, an approximation that seems to agree with a self-consistent density matrix calculation. I then look at the transmission functions in section~\ref{sec:twositetransmission}. The $I(V)$ characteristics  are presented in stability diagrams in section~\ref{sec:twositeparamsweep}. The improvement on quantitative agreement with the experiment of Ref.~\cite{perrinnano} is shown in section~\ref{sec:perrin}. Finally, I present a single result that makes use of the self-consistency ( equation~\ref{eq:selfconsistency} ) in section~\ref{sec:resultsselfconsistencycalc} and seems to improve on the agreement with the measurements by Ref.~\cite{perrinnano}.


\section{Two Site Model} 
The model comes in two flavours; with and without spin. The Hamiltonians, molecule-lead coupling matrices and capacitive self-energies are presented in the following subsections.
\label{sec:twosite}
\subsection{Spinless Two Site Model}
The form of the two-site model including Stark effect has been confirmed with DFT calculations in the supplement of Ref.~\cite{perrinnano}. I  will first look at the model without including spin. The Hamiltonian including tunnelling terms is:
\begin{align}
H_1 &= \begin{bmatrix} \epsilon_0 + \frac{1}{2} \alpha V & -\tau \\
-\tau & \epsilon_0 - \frac{1}{2} \alpha V\end{bmatrix},
\label{eq:spinlesshamiltonian}
\end{align}
where $\epsilon_0$ is the zero-bias level, $\tau$ is the tunnelling strength, $\alpha$ is the bias-level coupling due to the Stark effect and $V$ is the bias Voltage (in eV). The molecule is assumed to couple symmetrically to the left and right leads in the WBL:
\begin{align*}
\Gamma^L &= \begin{bmatrix} \Gamma & 0 \\ 0 & 0 \end{bmatrix},\\ \Gamma^R &= \begin{bmatrix} 0 & 0 \\ 0 & \Gamma \end{bmatrix},
\end{align*}
where $\Gamma$ is the molecule-lead coupling strength. The capacitive self-energy (equation~\ref{eq:selfenergycapacitive}) is given by:
\begin{align*}
\Sigma^c &= \begin{bmatrix} U & 0 \\ 0 & 0 \end{bmatrix} n_2 + \begin{bmatrix} 0 & 0 \\ 0 & U \end{bmatrix} n_1,
\end{align*}
where $U$ is the capacitive interaction strength.  The model is depicted schematically in Figure~\ref{fig:twosite}, where I have explicitly drawn the capacitive interaction. The leads are at different heights due to applied bias. In this model, where capacitive interaction is between two electrons instead of the electrons and the entire molecular background, the capacitive interaction is just the Coulomb Interaction\footnote{As previously indicated in section~\ref{sec:mbgfno}, there are molecules for which the interaction term is slightly more complicated and we can directly get the results from a number of DFT calculations.}.

\vspace{4cm}
\begin{figure}[!h]
    \begin{subfigure}[b]{0.44\textwidth}
        \includegraphics[height=.18\textheight]{pdf/non_interacting_schematics.pdf}\caption{non\hyp{}interacting}\label{fig:twositea}
    \end{subfigure}
    ~
    \begin{subfigure}[!tb]{0.44\textwidth}\vspace{-4.0cm}
        \includegraphics[height=.18\textheight]{pdf/interacting_schematics.pdf}\caption{interacting}\label{fig:twositeb}
    \end{subfigure}
    \caption{Schematic pictures of the spinless two site model. The red dashed line denotes the Fermi-level. Figure~(a) shows the non\hyp{}interacting case, where $\Gamma$ is the molecule-lead coupling, $\epsilon \pm \frac{1}{2} \alpha V$ are the left (right) level and $\tau$ is the tunnelling strength between the levels. In Figure~(b), an electron (large green arrow) occupies the right level. The swirly line is meant to indicate an interaction with electrons on the left level, thereby raising the level with the capacitive interaction or charging energy $U$.} \label{fig:twosite}
\end{figure}
\clearpage
When no capacitive interaction is included (i.e. $U=0$), the common method (section~\ref{sec:synthesis}) can be applied to find the transmission analytically: \cite{perrinnano}\footnote{If the bias voltage is assumed to be distributed symmetrically over the leads, then the current can be found analytically as well. However, this serves no purpose in this discussion.}:
\begin{align*}
T(\epsilon) &= \frac{ (2\tau)^2 }{(\frac{\Gamma}{2})^2} \frac{(\frac{\Gamma}{2})^2}{(\epsilon-\epsilon_1)^2 + (\frac{\Gamma}{2})^2}\frac{(\frac{\Gamma}{2})^2}{(\epsilon-\epsilon_2)^2 + (\frac{\Gamma}{2})^2},
\end{align*}
where $\epsilon_{1,2} = \epsilon_0 \pm \frac{1}{2} \Delta$, and $\Delta$ is the level splitting in the presence of bias voltage given by $\Delta = \sqrt{ (\alpha V)^2+ 4\tau^2}$. 

For the two-site model including capacitive interactions, I assume the chance that a many-body state $\ket{\kappa}$ is occupied is proportional to the Boltzmann-factor $e^{ -\beta \braket{ \kappa\left| H \right| \kappa}} Z^{-1}$, where $Z$ is a normalisation constant called the `partition function' and $H$ the full Hamiltonian including capacitive interactions. This is an approximation to the full non-equilibrium density matrix which we discuss in section~\ref{sec:dmapprox}.

It is illustrative for the application of the many-body Green's function (equation~\ref{eq:mbgfresult}) to consider the shapes of $G^{\lambda\pm}$. Their most important contribution in this context is simply the thermal average of the capacitive self-energy $\braket{\lambda\left|\Sigma^c\right|\lambda}$. The four $G^{\lambda\pm}$ are:
\begin{align*}
G^{\ket{00}\pm} &= \left[ \epsilon \begin{bmatrix} 1 & 0 \\ 0 & 1 \end{bmatrix} - \begin{bmatrix} \epsilon_0 + \frac{1}{2} \alpha V & -\tau \\
-\tau & \epsilon_0 - \frac{1}{2} \alpha V\end{bmatrix}  \pm \frac{\imath}{2} \begin{pmatrix} \Gamma & 0 \\ 0 & \Gamma \end{pmatrix} \right]^{-1}, \\
G^{\ket{10}\pm} &= \left[ \epsilon \begin{bmatrix} 1 & 0 \\ 0 & 1 \end{bmatrix} - \begin{bmatrix} \epsilon_0 + \frac{1}{2} \alpha V & -\tau \\
-\tau & \epsilon_0 - \frac{1}{2} \alpha V\end{bmatrix} - \begin{pmatrix} 0 & 0 \\ 0 & U \end{pmatrix} \pm \frac{\imath}{2} \begin{pmatrix} \Gamma & 0 \\ 0 & \Gamma \end{pmatrix} \right]^{-1}, \\
G^{\ket{01}\pm} &= \left[ \epsilon \begin{bmatrix} 1 & 0 \\ 0 & 1 \end{bmatrix} - \begin{bmatrix} \epsilon_0 + \frac{1}{2} \alpha V & -\tau \\
-\tau & \epsilon_0 - \frac{1}{2} \alpha V\end{bmatrix} - \begin{pmatrix} U & 0 \\ 0 & 0 \end{pmatrix} \pm \frac{\imath}{2} \begin{pmatrix} \Gamma & 0 \\ 0 & \Gamma \end{pmatrix} \right]^{-1},\\
G^{\ket{11}\pm} &= \left[ \epsilon \begin{bmatrix} 1 & 0 \\ 0 & 1 \end{bmatrix} - \begin{bmatrix} \epsilon_0 + \frac{1}{2} \alpha V & -\tau \\
-\tau & \epsilon_0 - \frac{1}{2} \alpha V\end{bmatrix} - \begin{pmatrix} U & 0 \\ 0 & U \end{pmatrix} \pm \frac{\imath}{2} \begin{pmatrix} \Gamma & 0 \\ 0 & \Gamma \end{pmatrix} \right]^{-1},
\end{align*}
where it is known that adding an electron to e.g. $\ket{10}$ would add the energy $\epsilon_2$ and the capacitive interaction energy $U$, as is reflected in the capacitive self-energy contribution of $G^{\ket{10}\pm}$. 

\subsection{Spinfull Two Site Model}
By including spin in the model, it is essentially keeping two copies of the model with added interaction. There is no spin-flip tunnelling. I use the ordered many-body basis $\left\{ \ket{\uparrow 1}, \ket{\downarrow 1}, \ket{\uparrow 2}, \ket{\downarrow 2}\right\}$. The Hamiltonian is:
\begin{align}
H_1 &= \begin{bmatrix} \epsilon_0 + \frac{1}{2} \alpha V & 0 & -\tau & 0 \\ 0 & \epsilon_0 + \frac{1}{2} \alpha V & 0 & -\tau\\ -\tau & 0 & \epsilon_0 - \frac{1}{2} \alpha V & 0 \\ 0 & -\tau & 0 & \epsilon_0 - \frac{1}{2} \alpha V\end{bmatrix},
\label{eq:spinfullhamiltonian}
\end{align} 
while the coupling matrices are:
\begin{align*}
\Gamma^L &= \begin{bmatrix} \Gamma & 0 & 0 & 0 \\ 0 & \Gamma & 0 & 0 \\ 0 & 0 & 0 & 0 \\  0 & 0 & 0 & 0\end{bmatrix},\\ \Gamma^R &= \begin{bmatrix} 0 & 0 & 0 & 0 \\ 0 & 0 & 0 & 0 \\ 0 & 0 & \Gamma & 0 \\ 0 & 0 & 0 & \Gamma \\ \end{bmatrix},
\end{align*}
and the capacitive self-energy is:
\begin{align*}
\Sigma^c &= \begin{bmatrix} U & 0 & 0 & 0\\ 0 & U & 0 & 0\\ 0 & 0 & 0 & 0\\ 0 & 0 & 0 & \xi U \end{bmatrix} n_{\uparrow 2} + \begin{bmatrix} U & 0 & 0 & 0\\ 0 & U & 0 & 0\\ 0 & 0 & \xi U & 0\\ 0 & 0 & 0 & 0 \end{bmatrix} n_{\downarrow 2} +\\
&\quad\begin{bmatrix} 0 & 0 & 0 & 0\\ 0 & \xi  U & 0 & 0\\ 0 & 0 & U & 0\\ 0 & 0 & 0 & U \end{bmatrix} n_{\uparrow 1} + \begin{bmatrix} \xi  U & 0 & 0 & 0\\ 0 & 0 & 0 & 0\\ 0 & 0 & U & 0\\ 0 & 0 & 0 & U \end{bmatrix} n_{\downarrow 1},
\end{align*}
where $U$ describes the strength of capacitive interaction between the left and right site (`intersite'), whereas $\xi U$ describes the strength of capacitive interaction on the left or right site (`onsite').

The definitions for the coupling matrices $\Gamma^{R,L}$ allow us to directly calculate the transmission analytically in terms of the retarded and advanced Green's Function \emph{if only a single state is occupied}:
\begin{align*}
T(\epsilon) &= \Gamma^2 \left( G^+_{13} G^-_{31} + G^+_{14} G^-_{41} + G^+_{23} G^-_{32} + G^+_{24} G^-_{42} \right),
\end{align*}
where $\Gamma$ is the WBL coupling constant. Note that there is in principle a maximum of $4$ peaks. Because $\left(G^+\right)_{ij}^\star = G^-_{ji}$ in the energy-domain, there is no quantum interference and the function is real valued, as it should be. If the density matrix specifies multiple occupied states, the many-body character of $\mathscr{G}^\pm$ and quantum interference between many-body states is possible.

\section{Density Matrix}
\label{sec:dmapprox}
The self-consistent non-equilibrium density matrix is shown to agree reasonably with the Boltzmann distribution, which is adopted as the approximation for most results.
\subsection{Spinless Two Site Model}
In section~\ref{sec:twositetransmission}, I will look at transmission figures for selected parameters. The many-body character of the new theory makes it imperative to look at the non-equilibrium matrix before I can make predictions about the different figures.

The initial guess for the self-consistency procedure (equation~\ref{eq:selfconsistency}) starts with the Boltzmann distribution as an initial guess, which specifies that only the lowest-energy many-body state is occupied. However, due to time constraints I have not used the self-consistency procedure, only working with the initial guess.

I conferred with Jose Celis Gil, and we agreed that the low temperature energy scale compared to the energy scales of the single-particle levels would cause rapid dissipation, implying a near\hyp{}equilibrium state, thereby justifying the approximation to the many-body occupation probabilities. Nevertheless, I did check this assumption.Figure~\ref{fig:occprob} shows that the self-consistent occupation probabilities (coloured diamonds) differ from the Boltzmann distribution (dashed lines) for $\left|V\right|\lesssim 0.10$. The theoretical mismatch of peak current in Ref.~\cite{perrinnano} has peaks at $\left|V\right|\approx 0.05$, so that the mismatch in occupation probabilities is only significant when no transmission peaks aligns with the Fermi-level. The mismatch is small when the levels start aligning, but seems sufficiently small that approximating by the Boltzmann distribution seems justifiable. In section~\ref{sec:resultsselfconsistencycalc}, I present a result that makes use of the self-consistency equation.


Jose Celis Gil has also looked at DFT calculations for the different charge states of the AH molecule and provided me with an estimate of $U \approx 0.30$, which is the value adopted in section~\ref{sec:perrin} and section~\ref{sec:resultsselfconsistencycalc}. The other parameters are initially the values used in Ref.~\cite{perrinnano}, slightly varied to find good agreement with the experiment.

\begin{figure}[p]
    \centering
    \includegraphics[width=.95\textwidth]{pdf/selfconsistent_low_temperature_1_v2.pdf}
    \caption{Occupation probabilities $P_{\kappa}$ found by the self-consistency procedure (equation~\ref{eq:selfconsistency}) versus potential difference $V$ (coloured diamonds). The many-body occupation probabilities are $P_{00}$ (red), $P_{10}$ (green), $P_{01}$ (blue) and $P_{11}$ (magenta). The self-consistent results differ slightly from the Boltzmann distribution (dashed lines) around $V=0$, but do not differ from the Boltzmann distribution for $\left|V\right|\gtrsim 0.10$. The parameters used are $\tau=0.010$, $\Gamma=0.010$, $\alpha=0.40$, $U=0.300$, $\epsilon_0=0$. }
    \label{fig:occprob}
\end{figure} \clearpage

Figure~\ref{fig:perrinenergy1}, where capacitive interaction is weak ($U=0.05$), is of interest because it shows the many-body character of my calculation, because the capacitive interaction enables the switch from $\ket{01}$ to $\ket{11}$, which also causes a switch in transport behaviour. In the first state, there will be one transmission peak at $\epsilon_R$ and one at $\epsilon_L + U$. However, in the second state the second peak is also raised to $\epsilon_R+U$. Since this moves the levels further off-resonance, the transmission peak will be of lower height. 
\begin{figure}[bt]
    \centering
    \includegraphics[height=.45\textheight]{pdf/energy/perrin_distribution_u1.pdf}
    \caption{Energy of many-body states versus zero-bias level $\epsilon_0$ for the spinless two-site model at $U=0.05$. The energies are for the following many-body states: $E_{00}$ (red circle), $E_{10}$ (green diamond), $E_{01}$ (blue triangle) and $E_{11}$ (magenta inverted triangle). At $\epsilon_0 \approx 0$, there is one electron on the right level. Lowering $\epsilon_0$ past ~$-0.12$ indicates a shift to an occupation of both the left and right levels. In this figure, $\alpha=0.5$, $V=0.25$ and $U=0.050$.}
    \label{fig:perrinenergy1}
\end{figure} 

 
\subsection{Spinfull Two Site Model} 
In this section, I present the energies of the spinfull model for $U=0.05$ and  $\xi=1.00$, parameters that match the transmission figure presented in section~\ref{sec:twositetransmission}.
 
\begin{figure}[htb]
    \centering
    \includegraphics[height=.45\textheight]{pdf/energy/pespin_distribution_u1_k2.pdf}
    \caption{Energy of many-body states versus zero-bias level $\epsilon_0$ for the spinfull two-site model. In this figure, $\alpha=0.5$, $V=0.25$, $U=0.05$ and $\xi=1.00$. The lowest energy state is $\ket{0011}$ for $\epsilon_0\gtrsim -0.15$, which then switches to $\ket{1011}$ before switching to $\ket{1111}$ at $\epsilon_0 \lesssim -0.20$. The following many body states are shown: $E_{0000}$ (red circle),  $E_{0001}$ (red diamond),  $E_{0010}$ (red triangle),  $E_{0011}$ (red inverted triangle), $E_{0100}$ (green circle), $E_{0101}$ (green diamond), $E_{0110}$ (green triangle), $E_{0111}$ (green inverted triangle),  $E_{1000}$ (blue circle),  $E_{1001}$ (blue diamond), 
    $E_{1010}$ (blue triangle),  $E_{1011}$ (blue inverted triangle),  $E_{1100}$ (magenta circle),  $E_{1101}$ (magenta diamond), $E_{1110}$ (magenta triangle) and $E_{1111}$ (magenta inverted triangle). }
    \label{fig:perspinenergy12}
\end{figure}  


The Fock-space of occupation states for the spinfull model has dimension $2^4$, so that there are that many energies in the figures. In Figure~\ref{fig:perspinenergy12}, I see similar behaviour as in the spinless case (Figure~\ref{fig:perrinenergy1}). There are switches between different many-body states, which are expected to cause some transmission peaks to shift by $U$ or $\xi U$ or both. 

\clearpage\section{Transmission}
\label{sec:twositetransmission}
Transmissions for both flavours (spinless, spinfull) of the two-site model are presented. 
\subsection{Spinless Two Site Model}
First, I want to look at the transmission for the spinless model. Figure~\ref{fig:spinlesstransmission} compares an interacting ($U=0.5$) transmission spectrum versus a non-interacting transmission spectrum. The effect of interaction is apparently to suppress the transmission peak height and move the rightmost peak outwards by $U$. 
\begin{figure}[htb]
    \centering
    \includegraphics[height=.35\textheight,clip=true,trim=5cm 2cm 4cm 3cm]{pdf/isbetter.pdf}
    \caption{Non-interacting (dashed red) and interacting (green) transmission spectra for $\alpha=0.74$, $\tau=0.024$, $\Gamma=0.010$, $\epsilon_0 = 0$, $V=0.50$ and $U=0.5$. In comparison with the non-interacting transmission, the peaks have dropped significantly and the rightmost peak is exactly $U$ removed from the corresponding eigenvalue (black diamonds).}
    \label{fig:spinlesstransmission}
\end{figure}

The interacting transmission spectrum looks similar to a non-interacting transmission spectrum, but the peak values differ. It is perhaps illustrative to consider Figure~\ref{fig:perrin_effective}. In this figure the `effective parameters' change almost linearly with increasing $U$. The effective parameters are those parameters for which the features of the transmission agree with a non-interacting model ignoring peak height. Thus, these are the parameters if a non-interacting fit with a scaling constant are fitted to an interacting $I(V)$ such as apparently done in Ref.~\cite{perrinnano}.
\begin{figure}[htb]
    \centering
    \includegraphics[height=.35\textheight]{pdf/trustme.pdf}
    \caption{In this figure, $\epsilon_0$ and $\tau$ are for a non-interacting model that is fitted to an interacting model, leading to `effective' parameters. This ignores the fact that the peaks of an interacting transmission are suppressed. The effective parameters are mostly $\propto U$, with $\tau$ diverging slightly because for $V\gg U$ the bias dominates the energy-splitting $\Delta$ for low $U$. The other parameters in this figure are $\alpha=0.74$, $\epsilon_0 = -0.050$, $\Gamma = 0.031$ and $V=0.500$ .}
    \label{fig:perrin_effective}
\end{figure}


These figures indicate that a mismatch in peak values such as that found in Ref.~\cite{perrinnano} may well be explained in terms of capacitive interaction. The shape of the differential conductance would be fitting for a two-site model without capacitive interaction, but the amplitude element of capacitive interaction is merely seen as quantitative disagreement. This notion will be discussed further in section~\ref{sec:perrin}. 

\subsection{Spinfull Two Site Model}
I now want to look at the transmission of the spinfull model \emph{in comparison} with that of the spinless model. First, recall that $U$ describes the intersite interaction while $\xi U$ the onsite interaction. Given the large amount of possible pictures, here I present a differential conductance map, which is a map of bias and gate voltage coloured by the transmission $T(\epsilon)$, that is distinct from the spinless model and another with extremely strong onsite interaction $\xi$, such that the spinfull and spinless model are expected to converge. 
  
\begin{figure}[tb]
    \centering
    \includegraphics[height=.38\textheight]{pdf/map/transmap_u1_k2.pdf}
    \caption{A differential conductance map, showing the transmission (colour) versus energy $\epsilon$ and zero-bias level $\epsilon_0$. A distinct difference is very clear, with a three-peak transmission spectrum making a short appearance. The parameters for this figure are $\alpha=0.50$, $\tau=0.01$, $\gamma=0.01$, $V=0.25$ and $U=0.050$. The onsite-intersite ratio is $\xi=1.0$. These figures are for discrete vertical values, and their vertical broadening is a plotting artifact.}
    \label{fig:transmap12}
\end{figure} 
\begin{figure}[tb]
    \centering
    \includegraphics[height=.38\textheight]{pdf/map/transmap_u3_k4.pdf}
    \caption{A differential conductance map, showing the transmission (colour) versus energy $\epsilon$ and zero-bias level $\epsilon_0$. There is no difference between the spinless and spinfull model, because of the extremely large onsite interaction which causes the molecule to behave as if only electrons of spin equal to the inhabited site can pass, although the spinfull model is in a equal superposition of spin states. The parameters for this figure are $\alpha=0.50$, $\tau=0.01$, $\gamma=0.01$, $V=0.25$ and $U=0.050$. The onsite-intersite ratio is $\xi=20.0$. These figures are for discrete vertical values, and their vertical broadening is a plotting artifact.}
    \label{fig:transmap34}
\end{figure}

Evidence of the many-body character of both models is clear in Figure~\ref{fig:transmap12}, where there is one switch in transmission peaks for the spinless model and two switches for the spinfull model. The spinless model switches from $\ket{01}$ to $\ket{11}$ (Figure~\ref{fig:perrinenergy1}), displacing one peak by an amount equal to the capacitive interaction strength $U$, which agrees well with the transmission figure.  The spinfull model, on the other hand, switches from $\ket{0011}$ to either $\ket{0111}$ or $\ket{1011}$ before switching to $\ket{1111}$ (Figure~\ref{fig:perspinenergy12}). Note that the rightmost peak corresponds to the left site, and the leftmost peak corresponds to the right site. The first switch displaces the leftmost peak\footnote{Adding an electron to the left level adds $U$ to the right level, which displaces the left peak.} and seems to have placed a third peak at the rightmost peak plus  an amount equal to the capacitive interaction strength $U$, a likely result because the energy states are degenerate. The second switch then displaces the leftmost peak by $\xi U$. 
 
Figure~\ref{fig:transmap34} is for parameters chosen such that the onsite energy $\xi U$ dominates so that only a single copy of the spinless model is active at any time. Therefore, not only do the peaks in transmission happen at the same location, but they are also of the same height.

\newpage Both figures presented here confirm the behaviour noted for the spinless model, and should be able to calculate a NDC effect at lower peak current, thus providing better agreement with Ref.~\cite{perrinnano}.


\clearpage\section{Stability Diagrams}
\label{sec:twositeparamsweep}
A colour-coded current map of the bias and gate voltage is commonly called a stability diagram. The zero-bias level $\epsilon_0$ should approximately correspond to a gate voltage $V_g$. The most interesting stability diagrams are those that include the many-body character. The current through a single-molecule junction has a well known feature called the Coulomb diamond \cite{seldenthuis, perrin}, visible in stability diagrams. Each diamond corresponds to a different charge state. I only present stability diagrams for the spinless case, for which throughout $-0.5 < \epsilon_0 < 0$ the occupied many-body states are $\ket{01}$ and $\ket{11}$. Therefore, I expect a single diamond to be visible in the stability diagrams.  

\begin{figure}[htb]
    \centering
    \includegraphics[height=.38\textheight]{pdf/coulombd/current_map_u2.pdf}
    \caption{Stability diagram of the spinless model. The Coulomb diamond is clear and has width $U$. The height of the Coulomb diamond is approximately $0.4$ [eV]. The triangular area to the left of the diamond corresponds to the next charge state. The distinct lines correspond to the bias window increasing to encompass a second transmission peak. The colours are based on the $^{10}\text{log}\left|I(V)\right|$, so that the NDC is implied because $I(V) = -I(-V)$. In this figure, $\alpha=0.5$, $\tau=0.02$, $\Gamma = 0.01$ and $U=0.15$.}
    \label{fig:currentmap2}
\end{figure} 

Figures~\ref{fig:currentmap2} and ~\ref{fig:diamond50} show the onset of the Coulomb diamond and that its width exactly equals the capacitive interaction strength $U$. The left triangular area visible in most of the figures is the next charge state, which for the spinless model is the state $\ket{11}$, whereas the Coulomb diamond is associated with $\ket{01}$ (at positive bias). The second charge state $\ket{11}$ effectively has $\epsilon_0+U$ as the average of the two levels and the triangular area is the suppression of current because the levels do not align with the Fermi-level.

Next, I explore the height of the Coulomb diamond. Analytically, the corresponding charge state leads to the following Green's function (section~\ref{sec:twosite}):
\begin{align*}
G^{\ket{01}\pm} &= \left[ \epsilon \begin{bmatrix} 1 & 0 \\ 0 & 1 \end{bmatrix} - \begin{bmatrix} \epsilon_0 + \frac{1}{2} \alpha V & -\tau \\
-\tau & \epsilon_0 - \frac{1}{2} \alpha V\end{bmatrix} - \begin{pmatrix} U & 0 \\ 0 & 0 \end{pmatrix} \pm \frac{\imath}{2} \begin{pmatrix} \Gamma & 0 \\ 0 & \Gamma \end{pmatrix} \right]^{-1},
\end{align*}
so that the transmission peaks should be found at eigenvalues of the (extended) mole\-cule Hamiltonian  $\begin{bmatrix} \epsilon_0 + \frac{1}{2} \alpha V+U & -\tau \\
-\tau & \epsilon_0 - \frac{1}{2} \alpha V\end{bmatrix}$. The current reaches its maximum when the transmission peaks enter the integration window, which physically represents that the transmission peaks align with electron states in the leads that are occupied in one lead and free in the other, i.e. $f_L(\epsilon) - f_R(\epsilon) \neq 0$.

Section~\ref{sec:twositetransmission} showed that the transmission peak locations change linearly with $\epsilon_0$, at least within a single charge state. Consider a state which starts on resonance, i.e. a transmission peak is within the bias window immediately. In this scenario, the leftmost transmission peak is within the bias window. Upon lowering $\epsilon_0$, the two transmission peaks shift slightly to the left. At higher bias, the intersection of the leftmost peak with the bias window is still at a lower bias voltage than that of the rightmost peak, but at a larger bias voltage than at higher $\epsilon_0$. As the lowering continues, the two values slowly approach each other. This is the corner of the Coulomb diamond.

Next, the the rightmost peak has the lower bias voltage at which it intersects the bias window and this intersection value is shrinking. This continues until the point where the rightmost peak is completely inside the bias window. This closes the diamond. 

The expectation is that the next charge state is then available, and indeed I find that the energies dictate that the molecule now occupies the next charge state.

The rate at which the transmission lines move outward depends almost entirely on $\alpha$. While it is technically the level-splitting $\Delta$, $\alpha V$ is the only growing term in $\Delta$, so that it will soon dominate the level-splitting, leading to $\Delta \approx \alpha V$. As such, I expect that the height of the Coulomb diamond depends mostly on $\alpha$. 

Indeed, this is what I find in Figures~\ref{fig:diamond50} and ~\ref{fig:diamond75}. These figures also show the co-tunnelling current within the diamond clearly. Here, the symmetry dictates that $I(V)=-I(-V)$, so that the co-tunnelling current switches signs at zero-bias $V=0$. Furthermore, the Stark effect pulls the sites further from each other, which leads to a reduction of the co-tunnelling current that is just barely visible in the stability diagrams. 
\begin{figure}[htb]
    \centering
    \includegraphics[height=.38\textheight]{pdf/coulombd/current_map_diamond_alpha_05.pdf}
    \caption{Stability diagram of the spinless model. The Coulomb diamond is clear and has width $U$. The height of the Coulomb diamond is approximately $0.6$ [eV]. The triangular area to the left of the diamond corresponds to the next charge state. The distinct lines correspond to the bias window increasing to encompass a second transmission peak. The colours are based on the $^{10}\text{log}\left|I(V)\right|$, so that the NDC is implied because $I(V) = -I(-V)$. In this figure, $\alpha=0.5$, $\tau=0.02$, $\Gamma = 0.01$ and $U=0.35$.}
    \label{fig:diamond50}
\end{figure}
\begin{figure}[htb]
    \centering
    \includegraphics[height=.38\textheight]{pdf/coulombd/current_map_diamond_alpha_075.pdf}
    \caption{Stability diagram of the spinless model. The Coulomb diamond is clear and has width $U$. The height of the Coulomb diamond is approximately $1.4$ [eV]. The triangular area to the left of the diamond corresponds to the next charge state. The distinct lines correspond to the bias window increasing to encompass a second transmission peak. The colours are based on the $^{10}\text{log}\left|I(V)\right|$, so that the NDC is implied because $I(V) = -I(-V)$. In this figure, $\alpha=0.75$, $\tau=0.02$, $\Gamma = 0.01$ and $U=0.35$.}
    \label{fig:diamond75}
\end{figure}

\clearpage\section{Experimental Fit}
\label{sec:perrin}
On inspection of the peak values of current, when the many-body occupation is the first charge state, I found that the current was already less than it was in \citet{perrinnano} or the zero-charge state. In Figure~\ref{fig:imax}, I compare the peak current for the edge of the first and second charge states. The first charge state has the left and right sites at different energies, so that current is suppressed relative to the second charge state, which has both levels at the same energy.
\begin{figure}[bt]
    \centering
    \includegraphics[height=.38\textheight]{pdf/canyouread.pdf}
    \caption{In this figure, $\alpha=0.50$, $\tau=0.02$, $\Gamma=0.01$ and $\epsilon_0$ is either $0$ (red) or $-U$ (green). I show the maximum current versus the capacitive interaction, for the first charge state (red diamonds) and the second charge state (green triangles). Visible is that the second charge state has a current that is within one order of magnitude larger, likely because the left/right levels align. For both sets, an inverse second-order polynomial of the capacitive interaction strength $U$ is a good fit.}
    \label{fig:imax}
\end{figure}

Figure~\ref{fig:imax} seems to indicate that the maximum current will flatten out, so that peak currents of approximately $10$ nA in the first charge state and approximately $5$ nA are expected for strong interaction $U \gtrsim 0.30$. This can be understood from the following. In the supplement to Ref.~\cite{perrinnano}, they compute the current for a two-site model in terms of the bonding ($\pi$) and anti-bonding ($\pi^\star$) orbitals. The current is proportional to the inverse of the level splitting, $\left( \Delta^2 + \Gamma^2 \right)$. The level splitting for the interacting case can be found from the eigenvalues of:
\begin{align*}
    \epsilon_{1,2} &= \text{eig}\begin{bmatrix} \epsilon_0 + \frac{1}{2} \alpha V+U & -\tau \\
-\tau & \epsilon_0 - \frac{1}{2} \alpha V\end{bmatrix} \\
&= \frac{2\epsilon_0 + U}{2} \pm \frac{1}{2} \Delta,
\end{align*} where $\Delta$ is defined via:
\begin{align*}
\Delta^2 &= U^2 + U \left(4\epsilon_0 - 2 \alpha V\right) + \alpha^2 V^2 - 4 \tau^2,
\end{align*}
so that the maximum current is proportional to the inverse of a second-order polynomial of capacitive interaction strength $U$. The fits in Figure~\ref{fig:imax} show that this simple analysis yields the correct results. The second charge state is expected to yield a higher current, as $\Delta$ reduces by $4 \epsilon_0 U$. This agrees with the concept that the first charge state has misaligned levels, because that indicates a larger $\Delta$ and thus a lower current, whereas the second charge state has a small $\Delta$ and a high current. However, I must caution that this analysis and Figure~\ref{fig:imax} concern only non-degenerate states.

These results indicate that the Coulomb interaction energy is in principle able to tune the peak current. Note that the Coulomb interaction energy will at some point induce a switch to a different many-body state (section~\ref{sec:dmapprox}). I will look at the measurements by Mickael Perrin \cite{perrin, perrinnano}, whom has kindly given me one of his datasets to look at. 

In figure~\ref{fig:perrindata}, a break junction measurement for different separation differences (gap sizes) is shown. As the separation distance increases, the current drops, until an apparent Coulomb gap opens suddenly around separation $s \approx 663$. As I showed in section~\ref{sec:dmapprox}, very sudden transmission changes can occur due to many-body state switches. I surmise a similar thing has happened here. As separation distance increases, we see that the region of current suppression around $V=0$ starts increasing in width. At approximately $s\approx 663$, it suddenly widens and a rather clear Coulomb blockade is visible. While there is no good way to estimate the theoretical parameters from such an experiment, speculation is possible. If all parameters change slowly, then this could simultaneously explain both the peak current dropping and the opening of a flat area in between. For $s \lesssim 663$, the zero charge state is likely occupied, based on the current at $s = 638$ and the smooth change from there. The zero bias level $\epsilon_0$ slowly moves off-resonance as separation increases, which is the opening of the plateau. Then, at $s \approx 663$, a many-body switch happens and the molecule is found in the first charge state. As the zero bias-level is not at the edge of a charge state, i.e. $\epsilon_0 \neq 0 $ and $\epsilon_0 \neq -U$, the model is inside the Coulomb diamond of the first charge state. Afterwards, the theme of slowly changing parameters continues and the gap widens. I must point out that in the stability diagram presented here, many-body switches always occurred at the tip of a diamond. However, I do think this speculation has merit.
\begin{figure}[htb]
    \centering
    \includegraphics[width=.99\textwidth,clip=true, trim=4cm 0cm 10cm 4cm]{pdf/perrin_experiment_abs.pdf}
    \caption{Experimental data received from Dr. Mickael Perrin of a breaking sequence in the experiments for Ref.~\citet{perrinnano}. The colours are proportional to $^{10}\text{log}\left|I(V)\right|$. As elsewhere, $I(V) = -I(-V)$. The current is capped at the lower end to $10^{-10}$ A.}
    \label{fig:perrindata}
\end{figure}

Next, I want to present a few results that show that the new theory is in better quantitative agreement with the results by Ref.~\cite{perrinnano}. However, the current $I(V)$ landscape at different parameters does not change smoothly, which makes it troublesome to find a converging fitting procedure. Instead, I performed an automatic rough parameter scan and selected results manually.

In particular, I have taken care to select results both in the first and second charge state, to show that both possibilities are relevant. First, I want to discuss the spinless result and then the spinfull result. For these results, $\alpha=0.400$, $\Gamma=0.010$, $\tau=0.010$ and $U=0.30$. The experimental current is measured in the exact same way as Figure~\ref{fig:perrindata}, but I show the average of the current in the region with suppressed current, but outside of the Coulomb blockade.

Figure~\ref{fig:fitspinless5} with $U=0.30$ and $\epsilon_0=-0.30$, shows a very clear NDC with maximum current within a factor of two relative to the averaged experimental current, a significant improvement over the previous prefactor $10^{-5}$ in Ref.~\cite{perrinnano}.


\begin{figure}[htb]
    \centering
    \includegraphics[width=.95\textwidth, clip=true, trim=11cm 2cm 2cm 0cm]{pdf/fit/fit_spinless_5.pdf}
    \caption{Experimental $I(V)$ (Magenta) versus calculated spinless current $I(V)$ (green). The current is scaled by $0.37$ so that it has the same peak height as the experimental current. In this figure, $\epsilon_0=-0.30$, $\tau=0.010$, $\Gamma=0.010$, $\alpha=0.400$ and $U=0.300$.}
    \label{fig:fitspinless5}
\end{figure} 
\begin{figure}[htb]
    \centering
    \includegraphics[width=.95\textwidth, clip=true, trim=11cm 2cm 2cm 0cm]{pdf/fit/fit_spinfull_1.pdf}
    \caption{Experimental $I(V)$ (Magenta) versus calculated spinfull current $I(V)$ (blue). The current is scaled by $0.88$ so that it has the same peak height as the experimental current. In this figure, $\epsilon_0=-0.30$, $\tau=0.010$, $\Gamma=0.010$, $\alpha=0.400$, $\xi=1.00$ and $U=0.300$.
    }
    \label{fig:fitspinfull1}
\end{figure} 

The second charge state is shown in Figure~\ref{fig:fitspinfull1}, which does have lower peak current, within a factor 2 of the experimental average. For $U=0.50$, the first charge state is very similar to the second, but is at $169$ \% while the second charge state is at $95$ \% of the experimental average.

To conclude, the quantitative agreement with \citet{perrinnano} is significantly improved in the first and second charge states. While the peak voltages do not exactly agree, I only performed a very limited rough scan of the parameter space. I am confident that the parameters can be tuned for better qualitative agreement, but could not do so due to time-constraints.

\clearpage\section{Self-Consistent Results}
\label{sec:resultsselfconsistencycalc}

In Figure~\ref{fig:occprob}, I showed that the self-consistent calculation for the non-equilibrium density matrix (equation~\ref{eq:selfconsistency}) agreed reasonably with the Boltzmann distribution. Due to time-constraints, I have not pursued self-consistent results throughout this thesis. Nevertheless, I wanted to include one spinless $I(V)$ found by the self-consistent approach, with parameters otherwise the same as in Figure~\ref{fig:fitspinless5}.

Figure~\ref{fig:figselfconsistent} shows the results of the self-consistent calculation versus the experimental average of $I(V)$. Apparently, the non-equilibrium density matrix pushes the current peaks slightly more outward, providing better qualitative agreement as well. The peak current was not changed much, so that the quantitative agreement is still extremely improved compared to Ref.~\cite{perrinnano}.


\begin{figure}[htb]
    \centering
    \includegraphics[width=.95\textwidth, clip=true, trim=11cm 2cm 2cm 0cm]{pdf/selfconsistent_fit_current_0.pdf}
    \caption{The interacting spinless current (red diamonds) with a density matrix calculated by means of the self-consistent approach (equation~\ref{eq:selfconsistency}) versus experimental data (magenta). In this figure, the parameters are $\alpha=0.40$, $\tau=0.010$, $\gamma=0.010$, $\epsilon_0=-0.300$ and $U=0.300$. The calculated current was scaled by a factor of $0.30$ so that it has the same peak height as the experimental current.  }  
    \label{fig:figselfconsistent}
\end{figure} 
%clearpage dumps all images in the stack. Also prevents images from skipping chapters.
\clearpage
\references{dissertation}