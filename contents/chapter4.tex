\chapter{Results}
\label{ch:chapter_4}

%% The following annotation is customary for chapter which have already been
%% published as a paper.
%\blfootnote{Parts of this chapter have been published in Annalen der Physik \textbf{324}, 289 (1906) \cite{Einstein1906}.}

%% It is only necessary to list the authors if multiple people contributed
%% significantly to the chapter.
%\authors{Albert {\titleshape Einstein}}

%% The '0pt' option ensures that no extra vertical space follows this epigraph,
%% since there is another epigraph after it.
%\epigraph[0pt]{
%    "quote1"
%}{attribution}

\epigraph{
    “No amount of experimentation can ever prove me right;
    
    a single experiment can prove me wrong. ”
}{Albert Einstein}

\begin{abstract}
Is this abstract enough?
\end{abstract}

%% Start the actual chapter on a new page.
\newpage
\section{Introduction}
While I have found the rather elegant many-body Green's Function (equation~\ref{eq:mbgfresult}), this does not tell us what the exact predictions are as they largely depend on the particular system or molecule.

In \citet{perrinnano}, the authors found a pronounced Negative-Differential Conductance (NDC) which is readily explained from a two site model, which I present here. Note that while the NDC was qualitatively explained by the two-site model, they used a prefactor $7.2 \times 10^{-5}$ to match the absolute values of the current. This has yet to be explained, and the Coulomb interaction was our prime suspect.

In section~\ref{sec:twosite}, I define the model both with and without spin. I then move on to look at the transmission functions in section~\ref{sec:twositetransmission}. The $I(V)$ characteristics for selected parameter sweeps are presented in section~\ref{sec:twositeparamsweep}. Finally, the improvement on quantitative agreement with the experiment of Ref.~\cite{perrinnano} is shown in section~\ref{sec:perrin}.


\section{Two site model} 
\label{sec:twosite}
The form of the two-site model including Stark effect has been confirmed with DFT calculations in the supplement of Ref.~\cite{perrinnano}. I  will first look at the model without including spin. The Hamiltonian including tunnelling terms is:
\begin{align}
H_1 &= \begin{bmatrix} \epsilon_0 + \frac{1}{2} \alpha V & -\tau \\
-\tau & \epsilon_0 - \frac{1}{2} \alpha V\end{bmatrix},
\label{eq:spinlesshamiltonian}
\end{align}
where $\epsilon_0$ is the zero-bias level, $\tau$ is the tunnelling strength, $\alpha$ is the bias-level coupling due to the Stark effect and $V$ is the bias Voltage (in eV). The molecule is assumed to couple symmetrically to the left and right leads in the WBL:
\begin{align*}
\Gamma^L &= \begin{bmatrix} \Gamma & 0 \\ 0 & 0 \end{bmatrix},\\ \Gamma^R &= \begin{bmatrix} 0 & 0 \\ 0 & \Gamma \end{bmatrix},
\end{align*}
where $\Gamma$ is the molecule-lead coupling strength. The capacitive self-energy (equation~\ref{eq:selfenergycapacitive}) is given by:
\begin{align*}
\Sigma^c &= \begin{bmatrix} U & 0 \\ 0 & 0 \end{bmatrix} n_2 + \begin{bmatrix} 0 & 0 \\ 0 & U \end{bmatrix} n_1,
\end{align*}
where $U$ is the capacitive interaction strength.
{\color{red} schematic depictions of system, fig a being 00 mb-state, fig b being 10 mb-state.}

When no capacitive interaction is included (i.e. $U=0$), the common method (section~\ref{sec:synthesis}) can be applied to find the transmission analytically: \cite{perrinnano}\footnote{If the bias voltage is assumed to be distributed symmetrically over the leads, then the current can be found analytically as well. However, this serves no purpose in this discussion.}:
\begin{align*}
T(\epsilon) &= \frac{ (2\tau)^2 }{(\frac{\Gamma}{2})^2} \frac{(\frac{\Gamma}{2})^2}{(\epsilon-\epsilon_1)^2 + (\frac{\Gamma}{2})^2}\frac{(\frac{\Gamma}{2})^2}{(\epsilon-\epsilon_2)^2 + (\frac{\Gamma}{2})^2},
\end{align*}
where $\epsilon_{1,2} = \epsilon_0 \pm \frac{1}{2} \Delta$, where $\Delta$ is the level splitting in the presence of bias voltage given by $\Delta = \sqrt{ (\alpha V)^2+ 4\tau^2}$. 

For the two-site model including capacitive interactions, we assume that the chance a many-body state $\ket{\kappa}$ is occupied is proportional to the Boltzmann-factor $e^{ -\beta \braket{ \kappa\left| H \right| \kappa}} Z^{-1}$, where $Z$ is a normalisation constant and $H$ the full Hamiltonian including capacitive interactions.

When we include spin in the model, we are essentially keeping two copies of the model and add interaction. There is no spin-flip tunnelling. I use the ordered many-body basis $\left\{ \ket{\uparrow 1}, \ket{\downarrow 1}, \ket{\uparrow 2}, \ket{\downarrow 2}\right\}$. The Hamiltonian is:
\begin{align}
H_1 &= \begin{bmatrix} \epsilon_0 + \frac{1}{2} \alpha V & 0 & -\tau & 0 \\ 0 & \epsilon_0 + \frac{1}{2} \alpha V & 0 & -\tau\\ -\tau & 0 & \epsilon_0 - \frac{1}{2} \alpha V & 0 \\ 0 & -\tau & 0 & \epsilon_0 - \frac{1}{2} \alpha V\end{bmatrix},
\label{eq:spinfullhamiltonian}
\end{align} 
while the coupling matrices are:
\begin{align*}
\Gamma^L &= \begin{bmatrix} \Gamma & 0 & 0 & 0 \\ 0 & \Gamma & 0 & 0 \\ 0 & 0 & 0 & 0 \\  0 & 0 & 0 & 0\end{bmatrix},\\ \Gamma^R &= \begin{bmatrix} 0 & 0 & 0 & 0 \\ 0 & 0 & 0 & 0 \\ 0 & 0 & \Gamma & 0 \\ 0 & 0 & 0 & \Gamma \\ \end{bmatrix},
\end{align*}
and the capacitive self-energy is:
\begin{align*}
\Sigma^c &= \begin{bmatrix} \zeta U & 0 & 0 & 0\\ 0 & \zeta U & 0 & 0\\ 0 & 0 & 0 & 0\\ 0 & 0 & 0 & \xi U \end{bmatrix} n_{\uparrow 2} + \begin{bmatrix} \zeta U & 0 & 0 & 0\\ 0 & \zeta U & 0 & 0\\ 0 & 0 & \xi U & 0\\ 0 & 0 & 0 & 0 \end{bmatrix} n_{\downarrow 2} +\\
&\quad\begin{bmatrix} 0 & 0 & 0 & 0\\ 0 & \xi  U & 0 & 0\\ 0 & 0 & \zeta U & 0\\ 0 & 0 & 0 & \zeta U \end{bmatrix} n_{\uparrow 1} + \begin{bmatrix} \xi  U & 0 & 0 & 0\\ 0 & 0 & 0 & 0\\ 0 & 0 & \zeta U & 0\\ 0 & 0 & 0 & \zeta U \end{bmatrix} n_{\downarrow 1},
\end{align*}
where $\zeta U$ describes the strength of capacitive interaction between the left and right site (intersite), whereas $\xi U$ describes the strength of capacitive interaction on the left or right site (onsite).

{\color{red} Pictures spinfull model}
\section{Transmission}
\label{sec:twositetransmission}
\section{Current parameter sweeps}
\label{sec:twositeparamsweep}
\section{Experimental fit}
\label{sec:perrin}
\references{dissertation}