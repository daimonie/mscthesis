\chapter{Summary}
\label{ch:chapter_5}

%% The following annotation is customary for chapter which have already been
%% published as a paper.
%\blfootnote{Parts of this chapter have been published in Annalen der Physik \textbf{324}, 289 (1906) \cite{Einstein1906}.}

%% It is only necessary to list the authors if multiple people contributed
%% significantly to the chapter.
%\authors{Albert {\titleshape Einstein}}

%% The '0pt' option ensures that no extra vertical space follows this epigraph,
%% since there is another epigraph after it.
%\epigraph[0pt]{
%    "quote1"
%}{attribution}

%no abstract

%% Start the actual chapter on a new page.
\newpage
\section{Introduction}
Most theses do not accurately depict all of the work performed. Rather, it depicts the successful work and does not mention the failures. For instance, we were quite excited about a region of apparent resonance within the Coulomb diamond, only to find out that this a plotting artifact in those pictures. Perhaps one of the greater possible improvements in science would the a journal called "The international journal of complete blunders". A great amount of resources is spent in futility, simply because scientists do not often communicate what did not work.

Even so, I think it is fair to say that the results as presented here are fairly consistent with intuition, theory and experiment. It gives a reasonable impression of the work done, both in derivation as in finding meaningful results in the two-site model. 

I have presented the main results, which I will summarise in section~\ref{sec:summary}. I then discuss the results and the implications of my thesis in section~\ref{sec:discussion}. Finally, I provide an outlook in section~\ref{sec:outlook}.


\section{Summary}
\label{sec:summary}
A brief overview of the experimental and theoretical status of the field was provided in chapter~\ref{ch:chapter_1}. In this chapter I listed a few of the experimental difficulties, outlined the ME approach and the difficulty of treating capacitive interaction.

A derivation of the non-equilibrium Green's Function Formalism was presented in chapter~\ref{ch:chapter_2}. I started with the Dyson equation (equation~\ref{eq:dyson}) and we used the Langreth rules to derive the Keldysh equation (equation~\ref{eq:keldysh}). The equation of motion (EOM) method for finding the exact shape of the Green's functions was demonstrated. I then showed how to calculate the spectral density, and more importantly the current by means of the Landauer equation (equation~\ref{eq:landauer}). I concluded that chapter by a short summary (section~\ref{sec:synthesis}).

In chapter~\ref{ch:chapter_3}, I looked at interactions and how to incorporate them to the non\hyp{}equilibrium Green's Function Formalism. In particular, I looked at electron-electron capacitive interaction and I presented Dr. Jos Seldenthuis his derivation of the capacitive self-energy and the resulting many-body Green's Function $\mathscr{G}^\kappa$. I then provided an alternative, simpler derivation of its expectation value.Noting the difficulty in finding the form of the non-equilibrium density matrix, I presented a derivation for a self-consistent scheme to find the non-equilibrium density matrix. As it turned out, the expectation value approach for the many-body Green's function was also useful in deriving an elaborate expression for the incorporation of interaction with a single bosonic level (vibrations, phonons). 

I then showed the explicit form of the capacitive-self energy for a two-site model, both with and without spin, in chapter~\ref{ch:chapter_4}. I briefly considered the self-consistent density matrix, but applied the equilibrium density matrix approximation throughout most results. I then presented results for the transmission which showed direct evidence of the many-body behaviour in our model. Next, I presented stability diagrams and explored the properties of the resulting Coulomb Diamonds. 

Finally, I revisited the experiment of Ref.~\cite{perrinnano}. The quantitative agreement was shown to have significantly improved. A brief look at a self-consistent $I(V)$ indicated that the self-consistent calculation can also improve the qualitative agreement.

\section{Discussion}
\label{sec:discussion}
Incorporating the capacitive interaction in the non-equilibrium Green's Function Formalism led to very clear Coulomb diamonds, which have been observed in experiments very often. Usually, these discussed in terms of charging effects \cite{seldenthuis, thijszantrev}, although simple systems have been discussed within the non-equilibrium Green's Function formalism \cite{haugjauho}. However, the derivation presented in this thesis includes capacitive interactions into the non-equilibrium Green's Function formalism analytically.

However, such many-body effects depend on the non-equilibrium density matrix, which is usually troublesome to find. I presented a self-consistent scheme to find the non-equilibrium density matrix, although I could not consistently use this approach due to time-constraints. 

The quantitative agreement of the new results with Ref.~\cite{perrinnano} is excellent for the parameters suggest by DFT calculations performed by Jose Celis Gil.  Both parameter-tuning and the self-consistent approach should lead to better qualitative agreement, which should conclude the theoretical analysis of the AH molecule.

\section{Future Outlook}
\label{sec:outlook}
A continuation of my thesis would be to investigate OPE3 \cite{frisenda}, which does not obey a simple toy-model as I presented here. However, as I pointed out for equation~\ref{eq:result}, the expression includes the effective single-particle Hamiltonian $H^\kappa = \mu^\kappa + \tau$ in a specific charge state $\ket{\kappa}$, which can be obtained from DFT calculations. Such an investigation would be very interesting, showing a novel direct application of DFT Hamiltonians within the non-equilibrium Green's Function formalism. Likewise, his findings suggest that the non-equilibrium density matrix will be extremely important, so that treatment of the OPE3 molecule includes both the many-body Green's function and the self-consistent approach for the non-equilibrium density matrix nicely.

With the plethora of molecules available for experiments, there is no doubt that toy-model treatments such as I performed or treatments similar to that suggested above for OPE3 will be very relevant.


%clearpage dumps all images in the stack. Also prevents images from skipping chapters.
\clearpage
\references{dissertation}